不久之前,金州勇士队通过分析大量的NBA赛事数据和观看大量的球队比赛,找到了防守安东尼·戴维斯的最佳球员。在最新的一赛季比赛中,勇士队的表现有了新的飞跃和突破。今年,几乎NBA每一支球队都开始追踪其球员的比赛数据,包括运动员在场上的得分位置,或者每个队员的比赛数据等。通过有效的数据分析,高级的数据模型,和分析工具,NBA比赛已经逐渐转化为专业篮球比赛。这样的转变不仅影响着球员的打球习惯,教练的训练方法,粉丝与球星的互动方式,甚至在赛制上也有了不同的调整运动员,教练,粉丝,甚至NBA分析员都希望好好利用大量的数据,来满足他们不同的需求。



篮球比赛在每个常规赛季中进行82场比赛。网上公开了大量的比赛数据,也有很多写得很好的文章和书籍分析球队球员表现.但是专业分析员会利用每场比赛球员的走位,和一些不对外公开的数据进行更具体详细的分析。由于数据的有限性,本位会利用网上公开的球员和球队数据进行分析,探索个体球员表现和球队表现的关系,本文最终目的是根据每队球员的具体数据预测NBA球队在季后赛的表现。


近几年,衡量球员表现的指标数据在近几年间不断更新,更加全面的反映球员的表现。最基本且最容易获得的数据是每个球员和球队每场比赛的得分数据。得分数据不仅记录某个球员和他队友的各项得分,还记录了他的对手球队和球员的防守数据。之前,得分数据仅仅包含最基础的球员数据,包括球员进球得分,篮板球,助攻数,和投篮命中率等。但这些数据来衡量一个球员的表现,能力和价值是远远不够的。首先,现有的统计量更多偏向记录球员的进攻数据,而忽视了球员的防守贡献水平;如果这个球员在整场比赛中投篮次数更多,助攻更多,这个球员可能是更有价值的球员,但尽管他有很强的的防守能力,比如他多次抢断对方投球,或盖帽,他也不一定能有很高的综合评分。其次,现有的统计数据可能对在场上拥有更多控球权的球员更有利;比如控球后卫将球带到前场并且更多的组织进攻,他带球时间更长,所以他的综合数据可能就更优秀。但是一个球员的控球能力却没有很精准的数据衡量,但控球能力是一个很重要的衡量指标。因此联盟从1970-1971年的赛季之后,开始记录球员的犯规数。


至今,由于有了更精准的记录仪器,我们可以获得更准确地球员数据,进行更具体的数据分析。例如金州勇士队,就大量利用大数据分析对球队整体训练计划和决策等方面进行改正。本文不会用到每一个球员在场上的实时位置信息和训练信息等十分具体的数据进行预测,因为这些数据不是对大众公开的。但是本文将会运用更常见的统计指标如射门得分,助攻次数,抢断次数,盖帽次数等,通过分析这些指标对球员和球队贡献的重要性来预测该球队的输赢。