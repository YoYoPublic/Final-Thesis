本文采用的是近2018-19年常规赛季的公开数据进行建模。由于比赛从赛制,到规则等各个方面都随着时代不断改变,甚至评判输赢的规则都有改变;尽管目前可以收集到自从1946-47赛季开始至今的所有赛事数据,之前的数据对现在比赛的分析也没有太多可借鉴作用。所以我们用今年来的所有常规赛事所公开的数据进行分析与建模。比如在上世纪末期,一个球队的输赢很大程度取决于个头最大的球员(如中风和大前锋),因为当时比赛的节奏相对较慢,主导比赛进程的球员通常是个头大的球员;但是如今,比赛节奏越来越快,比赛进程也更趋于数据化,专业化和技术化,这就为身形较小而灵活,但是技术水平很高,如投篮水平很高,控球能力很强的球员如史蒂芬库里,詹姆斯哈登,凯利欧文等身高不出众但命中率很高的球员。但联盟赛制转变后更有利于进攻型球员,尤其是擅长投射的球员,这使得比赛更激动人心,吸引更多观众的目光。\cite{fixler2012fading}
\begin{table}[h!]
	\begin{center}
		\begin{tabular}{ |c|c|c| } 
			\hline
			变量名 &	变量含义(英文)&	变量含义(中文)
\\
			\hline
			Rk	&Rank&	排名
\\
			Pos	&Position&	首发位置
\\
			Age	&Player's age&	年龄
\\
			Tm	&Team&	所在球队
\\
			G&	Games	&该赛季比赛总场数
\\
			MP&	Minutes Played&	球员上场时长
\\
			FG&	Field Goals	&球员投篮命中
\\
			FGA	&Field Goals Attempts&	球员投篮
\\
			FG\%	&Field Goal Percentage&	场均球员命中率
\\
			3P&	3-Point Field Goals	&场均3分球命中得分次数
\\
			3PA	&3-Point Field Goal Attempts&	场均3分球投篮次数
\\
			3P\%&	FG\% on 3-Pt FGAs&	场均3分球得分率
\\
			2P&	2-Point Field Goals	&场均2分球命中得分次数
\\
			2PA&	2-point Field Goal Attempts&	场均2分球投篮次数
\\
			2P\%&	FG\% on 2-Pt FGAs.&	场均2分球得分率
\\
			eFG\%&	Effective Field Goal Percent&	场均有效的投篮得分率
\\
			FT&	Free Throws	&场均罚球得分
\\
			FTA	&Free Throw Attempts&	场均罚球投篮次数
\\
			FT\%	&Free Throw Percentage&	场均罚球命中率
\\
			ORB	&Offensive Rebound	&场均进攻篮板球次数
\\
			DRB&	Defensive Rebounds	&场均防守篮板球次数
\\
			TRB	&Total Rebounds	&场均篮板球总次数
\\
			AST&	Assists	&场均助攻次数
\\
			STL	&Steals	&场均盖帽次数
\\
			BLK	&Blocks	&场均抢断次数
\\
			TOV	&Turnovers	&场均失误次数
\\
			PF&	Personal Fouls	&场均个人犯规次数
\\
			PTS&	Points	&场均得分
\\
			\hline
		\end{tabular}
		\caption{每个赛季所有球员的整体数据}
		\label{tab:1}
	\end{center}
\end{table}


\begin{table}[t]
	\centering
	\begin{tabular}{|c|c|c|}
		\hline
		变量名	&英文名称&	中文解释
\\
		\hline
		Rk&	Rank&	球队在赛季中的排名
\\
		W&	Wins&	球队整赛季赢过的比赛
\\
		W/L\%&	Win-Loss Percentage&	球队的胜负率
\\
		MOV	&Margin of Victory&	输赢球队比分之差
\\
		Ortg	&Offensive Rating&	每100次进攻的得分
\\
		DRtg&	Defensive Rating&	每100次进攻的失分
\\
		NRtg&	Net Rating&	每100次进攻机会的净胜分
\\
		MOV/A&	Adjusted Margin of Victory&	根据对手进攻节奏调整后的MOV
\\
		Ortg/A	&Adjusted Offensive Rating&	根据对手进攻节奏调整后的每100次进攻得分
\\
		DRtg/A&	Adjusted Defensive Rating	&根据对手进攻节奏调整后的失分
\\
		NRtg/A&	Adjusted Net Rating	&根据对手进攻节奏调整后的净胜分
\\
		\hline
	\end{tabular}
	\caption {每个赛季所有球员的整体数据}
	\label{2}
	
\end{table}


\begin{enumerate}
	\item  以2018-19赛季比赛中球员数据为例,如表\ref{tab:1}
	\item 2018-19赛季比赛中球队数据为例,如表\ref{2}
	\item 收集每个球队的整个赛季中与之对抗的球队的比赛数据,变量名如\ref{tab:1}。
	\item 收集每个球队在整个赛季中数据统计,变量名如\ref{tab:1}.

\end{enumerate}





